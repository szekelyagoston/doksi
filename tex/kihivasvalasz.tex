\section{Képfeldolgozó megoldások kihívás-válasz implementálására}
\subsection{Alapok}
A kihívás - válasz megoldások célja a felhasználó-hitelesítés és -azonosítás valósidejűségének ellenőrzése, nélküle a természetes és virtuális személy összehasonlítása után ( amennyiben a két képen található arc megfelelő valószínűséggel azonos emberhez tartozik ) , az authentikációt elfogadottnak vesszük.
Gondoljunk bele ennek a megoldásnak a sebezhetőségeibe. A legegyszerűbb feltörése a rendszernek, ha az elektronikus személyi igazolvánnyal történő azonosítás után a kamera elé helyezzük az igazolványt, képpel kamerának nézve. A használt képfeldolgozó szoftver nagy valószínűséggel fel fogja ismerni a képet, és magas százalékos valószínűséget fog dobni az összetartozóságra. Ennek a verziónak a sebezhetőségén kiívánunk javítani kihívás - válasz megoldás implementálásával.
\subsection{Kihívás - válasz első gondolatok}
Ennek a szakdolgozatnak célja nem különböző képfeldolgozó eszközök létrehozása, ami alapján ezeket a kihívásokat értékelni tudjuk. Leginkább publikusan elérhető third party lehetőségeket felkutatása volt a cél, amelyek önállóan tudnak feldolgozni képet vagy videót, majd bizonyos felismert tulajdonságokkal visszatérni a rendszerhez. Ezután ezeket a válaszokat feldolgozva eldönthető, hogy a bejelentkezést engedjük vagy megtagadjuk.
Alapvetően két típust különböztetünk meg elérhetőség szempontjából
\begin{itemize}
\item online
\item offline
\end{itemize}
Egy másik fajta osztályozás pedig működés szempontjából lehetséges
\begin{itemize}
\item implementálja a kihívás - válasz protokollt, így video-streamből egy igen/nem választ, esetleg egy százalékos valószínűséget ad
\item képfeldolgozó könyvtár / api, ami kép-et vár kérésben, válaszban pedig a ennek valamilyen felismert tulajdonságaival tér vissza. Ezután nekünk kell eldönteni hogy elfogadjuk-e az adott kihívásra érkezett választ
\end{itemize}

Az első komoly nehézség rögtön adja magát : amennyiben online megoldást választunk, az komoly adatforalommal, és emiatt késleltetéssel járhat, pedig egy authentikációnál a gyorsaság nem elhanyagolható. Másik gyorsasági tényező az adott könyvtár/API képfeldolgozási sebessége különböző méretű/sebességű képek esetén. Ezt a feldolgozási időt drasztikusan csökkenteni lehet a képek méretének (és ezzel minőségének csökkentésével), de ez esetben komoly kérdés marad a képfeldolgozás minősége. Másik megoldás lehet a képek szürkeárnyalatossá alakítása, ezzel a képfeldolgozás sikeressége lehetséges hogy kevésbé fog csökkenni mint egy sima tömörítésnél, de a méretet csökkentheti.
\\A képek mérete és ezzel a felhasznált sávszélesség, illetve sebesség két helyen számít. Először amikor a felhasználó eszköze (jelen esetben android okostelefon) rögzíti a videót, kiválaszt belőle valamennyi képkockát, majd továbbítja az azonosítást végző szervernek. Ezután pedig amikor ez a szerver továbbküldi egy third party alkalmazásnak. Jól látható, hogy a legkevesebb adatforgalommal az jár, ha már a felhasználó telefonján transzformációkat (tömörítés, szürkeárnyalatossá konvertálás) hajtunk végre. Ezzel a megközelítéssel a probléma viszont a felhasználói eszközök sokszínűsége: Különböző erősségű telefonok különböző gyorsasággal tudják ezeket az alapvetően költséges műveleteket végrehajtani, szélsőséges esetben el is fogyhat a telefon alól a memória, ami az alkalmazás leállásához is vezethet.
\\Kézenfekvőnek látszódhat a költséges műveletek szerver oldalra való átcsoportosítása, hiszen ott a teljesítményt sokal hatékonyabban tudjuk maximalizálni, monitorozni, végső esetben pedig több erőforrást adni. Ám ez esetben pedig egy erős minőségű kamerával készített pár másodperces videó komoly adatforgalommal járna, ami mobilnet esetében kifejezetten kerülendő.
Ezeket a kérdéseket mindenképp körül kell járni a tesztelés szakaszában, és a legoptimálisabbat megtalálni.
\\Összegezve  a fenti gondolatokat a következőket érdemes feljegyezni:

\begin{center}
	\begin{tabular}{|p{2cm}|p{3cm} |p{3cm} | p{3cm}|p{3cm}|}
   	\hline
	\textbf{Kép transzformációk helye} & \textbf{Sebesség} & \textbf{Adatforgalom} & \textbf{Képfeldolgozás minősége} &\textbf{ Hibalehetőségek} \\ \hline
	Nincs transzformáció & Adatforgalom sebessége & Legnagyobb & Legjobb & Kevés \\ \hline
	Felhasználó eszközén & Felhasználó eszközének minőségétől függ, plusz kétszeresen csökkentett adatforgalom sebessége & Legalacsonyabb & Képfeldolgozó szolgáltatás minőségétől függ & Ha gyenge a felhasználó eszköze, az alkalmazás nagyon lelassulhat, esetleg le is állhat \\ \hline
	Alkalmazás-szerveren & Egyszeresen csökkentett adatforgalom sebessége, plusz a szerver transzformációinak sebessége & Felhasználó szempontjából legnagyobb & Képfeldolgozó szolgáltatás minőségétől függ & Kevés\\ \hline
	\end{tabular}
\end{center}

A szakdolgozat keretein belül a második esetet, tehát a felhazsnáló eszközén történő transzformációt valósítottuk meg. A tesztelés feladata lesz a megfelelő paraméterek beállítása a képtömörítéshez, hogy a minimalizáljuk azt az adatmennyiséget illetve sebességet ami az adott képfeldolgozó eszköz megbízható működéséhez még szükséges.

\subsection{Képfeldolgozó eszközök}
Több online illetve offline képfeldolgozási lehetőséget vizsgáltunk meg, ezeknek egy része számunkra jelenleg nem használható, vagy használata nem fér bele a szakdolgozat kereteibe. A jelenlegi kihívás - válasz megoldásunk arc jobbra illetve balra fordulását keresi, illetve a két arc összetartozóságát állapítja meg, így elsősorban erre kihegyezve jegyeztük fel a talált lehetőségeket, de a jövőbeni fejlesztéseket szem előtt tartva más kihívás - válasz megoldásokra való használhatóságot is figyeltünk.
\subsubsection{Google Cloud Vision}
A Google Cloud Vision API \cite{GoogleCloudVision} A Google Cloud Platform Machine Learning moduljában található többek között a Cloud Speech API, és Cloud Translation API -val. Külön érdekes lehet számunkra hosszútávon a Cloud Video Itelligence API, ami jelenleg Private Beta státuszban van.
\\A számunkra érdekes Vision API egy képről az alábbi tulajdonságokat képes megállapítani:
\begin{itemize}
\item Különböző tárgyakat, élőlényeket, objektumokat ismer fel a képen és pozicionál
\item Tartalom moderálásra használható, szűrhet felnőtt, hamisított, erőszakos tartalomra vagy orvosi részleteket tartalmazó részekre (amik szintén rossz hatással lehetnek arra fel nem készült személyek számára)
\item Hasonló képeket keres az interneten, így például egy arcról eldönthet, hogy melyik celebre hasonlít a legjobban
\item Visszadja a felismert képeket minden tulajdonságukkal (szemöldök, száj pozíciók, stb.), illetve a felismert érzelmeket
\end{itemize}

Az utolsó pontot tudjuk mi ez esetben kihasználni, mégpedig a visszadaott arc pozícióját, ami a Roll, Tilt, Pan hármasból a Pan.
\\Nagyon fontos a különböző API-k vizsgálatakor, hogy mekkora elfordulásig ismerik fel az arcokat a képen. Első tesztnek 3 képet használtam, egy kamera felé nézőt, egy picit elfordítottat és egy nagyon elfordítottat (ebben az esetben csak 1 szem látszódik).
Az eredmények: \\

\begin{figure}[h]
 \begin{minipage}{.5\textwidth} 
    \includegraphics[scale=0.3]{img/cloud_vision_left}
    \caption{Balra néző kép, 25 fokos elfordulás}
 \end{minipage}
 \begin{minipage}{.5\textwidth} 
     \includegraphics[scale=0.3]{img/cloud_vision_very_left}
     \caption{Balra néző kép, 52 fokos elfordulás}
 \end{minipage}
\end{figure}

Vizsgáljuk meg külön mi történik a feldolgozás eredményével a komolyabb minőségromlás esetén. A feltöltendő képeket az eredeti ~500kb-ról ~40kb-ra csökkentettem, a minőségromlás szemmel látható. Hasonlítsuk össze az erre a két képre adott eredményt (3.ábra, és 4.ábra)
\begin{figure}[h]
 \begin{minipage}{.5\textwidth} 
    \includegraphics[scale=0.3]{img/cloud_vision_left_compressed}
    \caption{Balra néző tömörített kép, 25 fokos elfordulás}
 \end{minipage}
 \begin{minipage}{.5\textwidth} 
     \includegraphics[scale=0.3]{img/cloud_vision_very_left_compressed}
     \caption{Balra néző tömörített kép, 51 fokos elfordulás}
 \end{minipage}
\end{figure}

Jól látható, hogy jelentős méretcsökkentésre a feldolgozás megbízhatósága 95\%-ról 93\%-ra változott abban az esetben amikor a célszemély csak egyik szeme látszódik, a visszadaott tulajdonságok pedig nagyvonalakban megyegyeznek. 
Az eredmények láttán megállapítható, hogy a Google Cloud Vision széles skálán érzékeli az elfordulást, még akkor is, amikor a célszemély egyik szeme látszódik csak, és tömörítésre sem érzékeny.
\\A probléma, hogy önmagában ez az API kevés lenne az elvárt működés implemetálására, mivel arcösszehaonlítást nem tartalmaz. Így ha mégis a használat mellett döntünk, az összehasonlító modult egy másik eszközzel kell megoldani.

\subsubsection{IBM Watson}
A mesterséges intelligencia témakörében a legtöbb embernek először az IBM Watson ugrik be, úgyhogy érdemes lehet megnézni, hogy a képfeldolgozó modulja (Visual Recognition \cite{IBMWatson}) mit tud adni nekünk a mi témakörünkben. 
\begin{itemize}
\item megbízhatósági pontszámmal csoportokba sorolja a képeket, ami csoportosításokra használható (pl.: Person, female, bridesmaid)
\item felimseri az arcokat, és visszadja a pozíciókat, életkort tippel (minimum és maximum értékkel), illetve a felismert arc nemét állapítja meg.
\end{itemize}

A dokumentációban\cite{WATSON_DETECT} mélyebbre ásva sem lehet találni semmi olyan visszadaott értéket, ami az arcelforduláshoz segítséget adna, így megállapítható hogy nem lesz segítségünkre ez az API.

Arcösszehasonlítás szempontból is érdemes lehet megvizsgálni a Collections BETA\cite{WATSON_COLLECTIONS} verzióban lévő API-ját. A működési mechanizmus nagyon hasonló más collections típusú arcösszehasonlítást tartalmazó szolgáltatásokhoz. Először hozzáadjuk a képet egy általunk választott gyűjteményhez, majd egy már feltöltött képhez hasonlóakat tudunk keresni egy adott gyűjteményben. Ez a megoldás akár segíthet is arcösszehasonlításban, de ez esetben minden regisztráció után a képet egy új gyűjteményben kéne elhelyezni, majd ebben a gyűjteményben keresni később, ám nem erre lett elsősorban létrehozva ez a service. Az alapötlet az, hogy metadata adatokat lehet csatolni a feltöltött képekhez (pl.: név), így például egy visszatérő felhasználót név szerint tudnánk már szólítani. Ráerőltethető valamilyen szinten az összehasonlítás is, de elsősorban olyan szolgáltatást kéne keresni, ami két képről dönti el hogy egyeznek-e, és a képeket se kelljen tárolni harmadik félnél. Ezek alapján az IBM Watson ezen megoldását elvethetjük.


\newpage