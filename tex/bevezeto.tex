\section{Bevezetés}
A felhasználó azonosítás és hitelesítés minden érzékeny adathoz hozzáférő és felhasználó rendszer számára megkerülhetetlen. Az illetéktelen hozzáférés engedélyezése a legsúlyosabb sebezhetőségek közé tartozik, ami akár egy vállalkozást csődbe tud vinni.
A kiemelt érzékenységű adatok tárolására szakosodott rendszerek (pl. közigazgatás, banki rendzserek) a jelenlegi könnyen eltulajdonítható online megoldások helyett kénytelenek többnyire személyes ügyintézés útján kezelni a fontosabb felhasználói műveleteket.
\\
Vegyük példaként a széles körben elterjedt és használt egyfaktorú,  felhasználónév és jelszó alapú bejelentkezéseket. A felhasználó vagy az illetéktelen hozzáférést szerezni próbáló (nevezzük támadónak) bejelentkezést csak az különbözteti meg, hogy az adott felhasználónévhez tartozó jelszót ismeri-e a bejelentkezni próbáló személy. Amennyiben ez a jelszó megfelel, a rendszernek 
nincsen módja ellenőrizni hogy tényleg az eredeti felhasználó jelentkezett be vagy valaki más jutott hozzá a jelszóhoz. A fő problémát pedig az okozza, hogy egy jelszóhoz hozzáférni az esetek jó részében nem túl nehéz feladat.
\begin{itemize}
\item Egy átlag embernek akár több mint 10 különböző rendszerhez lehet hozzáférése. Sok jelszót senki sem szeret/tud megjegyezni, így ugyanazt fogja mindenhova használni vagy valahova feljegyzi magának oldalanként. Míg első esetben akár csak az egyik rendszerből való adatlopás esetén (külön veszély ha valahol nem biztonságosan tárolják a jelszavakat) az összes többi oldalhoz hozzáférést
 szerezhet a támadó, addig a második esetben ehhez a tároló fájlhoz való hozzájutás járhat hasonló következményekkel.
\item Jelszólopás történhet akár különböző keylogger programokkal, a bejelentkezés forgalmának lehallgatásával, a jelszó kitalálásával vagy akár vizuális úton (használat közben valaki leskelődik).
\item Külön sebezhetőség a gyakran használt vagy nem elég hoszzú jelszavak visszafejtése különböző jelszó-szótárak segítségével.
\end{itemize}
Jól látható hogy az ilyen alacsony megbízhatóságú módszerekre nem támaszkodhatnak fontos adatokat kezelő rendszerek.
\subsection{Jelenlegi megoldás hátrányai}
Megbízható online azonosítás hiányában az ilyen szintű biztonságot megkövetelő rendszerek a személyes megjelenést és ügyintézést vezették be. Hátrányai között szerepel a szolgáltató részéről az ügyfélszolgálati irodák fenntartása, a rendszerek üzemeltetése, az alkalmazottak fizetése, a felhasználók részéről pedig körülményes és nehezem teljesíthető lehet adott időben fizikailag egy adott helyen megjelenni. Ennél a megoldásnál mindenképpen egy gyorsabb, olcsóbb, legalább ugyanannyira megbízható és távolról is használható módszer mindkét fél számára pozitívumokkal járna.
\subsection{Egy megvalósítható megoldás}
Léteznek erős, több faktoros felhasználó-hitelesítési és -azonosítási megoldások, azonban a különböző jelszavak, egyszeri jelszavak, kriptográfiai megoldások mögött nem látható, hogy ki a felhasználó. Ezekben az esetekben az illetéktelen hozzáférés esélye lényegesen kisebb, de nincs lehetőség a felhasználó fizikai jelenlétének ellenőrzésére. 
\\Egy példa erre az OTP bank kétfaktoros azonosítás megoldása. Itt a megfelelő jelszó megadása után a felhasználó regisztrált telefonszámára érkezik egy SMS egy egyszeri jelszóval. A fentebb taglalt jelszólopási módok itt így nem működhetnek, a sikeres megtévesztéshez a felhasználó telefonjára is szükség van, amihez jóval nehezebb lehet hozzáférni. Ám ha sikerül, megnyílik az út a sikeres bejelentkezéshez.
\\A szakdolgozat keretében megvalósítandó felhasználó-hitelesítési és azonosítási megoldás először egy elektronikus személyi igazolvánnyal történő azonosítás keretében történik meg, majd a természetes és virtuális személy öszetartozóságát (hasonlóságát) ellenőrzi. A valósidejűségre pedig élőkép alapú kihívás-válasz protokoll adja a választ.
\\Végül ez a megoldás beépítésre kerül az E-Group IDX felhasználó-hitelesítési és azonosítási termékébe.