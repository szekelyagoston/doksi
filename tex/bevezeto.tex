\section{Feladatkiírás}
A szakdolgozat célja a felhasználó hitelesítési megoldás implementációja oly módon, hogy az elektronikus személyi igazolványt használó természetes személy jelenlétét, illetve a természetes személy és a virtuális személy összetartozóságát (hasonlóságát) is ellenőrizze.
\\Az összetartozóság ellenőrzésére az eSzemélyi igazolványból kiolvasott kép és a mobiltelefon kamerájával készített kép összehasonlításával kerül sor. Az összehasonlításhoz publikusan elérhető szolgáltatásokat és API-kat vizsgálunk meg (pl.: Microsoft Oxford Project), hasonlítunk össze és használunk fel.
\\Az eSzemélyi igazolványból történő adatkiolvasásra meg kell vizsgálni a már elérhető opensource lehetőségeket (pl. JMRTD), majd kiválasztani a használatra szánt API-t, vagy amennyiben ez nem elegendő, implementálni egy megoldást az adatok kiolvasására.
\\A jelenlét ellenőrzésére különböző élőkép feldolgozás alapú kihívás-válasz protokollok beépíthetőségét vizsgáljuk meg, illetve valósítjuk meg. Szintén publikus szolgáltatásokat keresünk, illetve különböző képfeldolgozó könyvtárakat vizsgálunk meg és hasonlítunk össze.
\newpage
\section{Bevezetés}
A felhasználó azonosítás és hitelesítés minden érzékeny adathoz hozzáférő és felhasználó rendszer számára megkerülhetetlen. Az illetéktelen hozzáférés engedélyezése a legsúlyosabb sebezhetőségek közé tartozik, ami akár egy vállalkozást csődbe tud vinni.
A kiemelt érzékenységű adatok tárolására szakosodott rendszerek (banki, közigazgatási) a jelenlegi könnyen eltulajdonítható online megoldások helyett kénytelenek többnyire személyes ügyintézés útján kezelni a fontosabb felhasználói műveleteket.
\\
Vegyük példaként a széles körben elterjedt és használt egyfaktorú,  felhasználónév és jelszó alapú bejelentkezéseket. A felhasználó vagy az illetéktelen hozzáférést szerezni próbáló (nevezzük támadónak) bejelentkezést csak az különbözteti meg, hogy az adott felhasználónévhez tartozó jelszót ismeri-e a bejelentkezni próbáló személy. Amennyiben ez a jelszó megfelel, a rendszernek 
nincsen módja ellenőrizni, hogy tényleg az eredeti felhasználó jelentkezett be vagy valaki más jutott hozzá a jelszóhoz. A fő problémát pedig az okozza, hogy egy jelszóhoz hozzáférni az esetek jó részében nem túl nehéz feladat.
\begin{itemize}
\item Egy átlag embernek akár több mint 10 különböző rendszerhez lehet hozzáférése. Sok jelszót senki sem szeret/tud megjegyezni, így ugyanazt fogja mindenhova használni vagy valahova feljegyzi magának oldalanként. Míg első esetben akár csak az egyik rendszerből való adatlopás esetén (külön veszély, ha valahol nem biztonságosan tárolják a jelszavakat) az összes többi oldalhoz hozzáférést
 szerezhet a támadó, addig a második esetben ehhez a tároló fájlhoz való hozzájutás járhat hasonló következményekkel.
\item Jelszólopás történhet akár különböző keylogger programokkal, a bejelentkezés forgalmának lehallgatásával, a jelszó kitalálásával vagy akár vizuális úton (használat közben valaki leskelődik).
\item Külön sebezhetőség a gyakran használt vagy nem elég hosszú jelszavak visszafejtése különböző jelszó-szótárak segítségével.
\end{itemize}
Jól látható hogy az ilyen alacsony megbízhatóságú módszerekre nem támaszkodhatnak fontos adatokat kezelő rendszerek.
\subsection{Jelenlegi módszerek hátrányai}
Megbízható online azonosítás hiányában az ilyen szintű biztonságot megkövetelő rendszerek a személyes megjelenést és ügyintézést vezették be. Hátrányai között szerepel a szolgáltató részéről az ügyfélszolgálati irodák fenntartása, a rendszerek üzemeltetése, az alkalmazottak fizetése, a felhasználók részéről pedig körülményes és nehezen teljesíthető lehet adott időben fizikailag egy adott helyen megjelenni. Ennél a megoldásnál mindenképpen egy gyorsabb, olcsóbb, legalább ugyanannyira megbízható és távolról is használható módszer mindkét fél számára pozitívumokkal járna.
\subsection{Egy megvalósítható megoldás}
Léteznek erős, több faktoros felhasználó-hitelesítési és -azonosítási megoldások, azonban a különböző jelszavak, egyszeri jelszavak, kriptográfiai megoldások mögött nem látható, hogy ki a felhasználó. Ezekben az esetekben az illetéktelen hozzáférés esélye lényegesen kisebb, de nincs lehetőség a felhasználó fizikai jelenlétének ellenőrzésére. 
\\Egy példa erre az OTP bank kétfaktoros azonosítás megoldása. Itt a megfelelő jelszó megadása után a felhasználó regisztrált telefonszámára érkezik egy SMS egy egyszeri jelszóval. A fentebb taglalt jelszólopási módok itt így nem működhetnek, a sikeres megtévesztéshez a felhasználó telefonjára is szükség van, amihez jóval nehezebb lehet hozzáférni. Ám ha sikerül, megnyílik az út a sikeres bejelentkezéshez.
\\A szakdolgozat keretében megvalósítandó felhasználó-hitelesítési és azonosítási megoldás először egy elektronikus személyi igazolvánnyal történő azonosítás keretében történik meg, majd a természetes és virtuális személy összetartozóságát (hasonlóságát) ellenőrzi. A valósidejűségre pedig élőkép alapú kihívás-válasz protokoll adja a választ.
\\Végül ez a megoldás beépítésre kerül az E-Group IDX felhasználó-hitelesítési és azonosítási termékébe.
\subsection{A megoldás lépései}
A következő pontokban a megoldás menetét bontom kisebb részekre az első gondolatoktól a kész szoftverig.
\begin{itemize}
\item Első lépésként annak néztem utána, hogy a célként kitűzött kihívás - válasz protokollt milyen külső eszközök segítségével lehet implementálni.  Kép és videófeldolgozó szolgáltatásokat kerestem és hasonlítottam össze, amelyek képesek elsősorban egy arcról megállapítani az elfordulás mértékét.
\item Ezután mivel videófeldolgozó szolgáltatást nem találtam megfelelőt, a képfeldolgozó eszközöket kezdtem közelebbről vizsgálni, tesztelni. A Google Mobile Vision, Microsoft Project Oxford és a Face++ szolgáltatását választottam vizsgálatra.
\item A könnyebb tesztelés érdekében egy külön Android tesztalkalmazást hoztam létre, amely egy a galériából kiválasztott képet továbbít valamely vizsgálandó szolgáltatás részére, az eredményt pedig a képernyőre kiírja.
\item Mindhárom esetben a visszaadott válaszokat megfelelőnek találtam további vizsgálatra, így elindultam még mindig a tesztalkalmazásban a videófeldolgozás felé. Első lépésben egy videóból kiválasztott egy darab képkockát elemeztem.
\item Miután ezt a tesztet is működőképesnek találtam a tesztalkalmazást végső állapotba fejlesztettem, mégpedig úgy, hogy egy videóból N darab képkockát választottam ki, majd továbbítottam. Ez alapján megállapítható volt, hogy az arcelfordulás alapú bejelentkezés feldolgozását megfelelő minőségben támogatják ezek a szolgáltatások.
\item A tesztalkalmazást félretéve a teljes rendszer összeállításába kezdtem bele. Létrehoztam egy weboldalt, amely bejelentkezési megoldásába integráltam az IDX szervert, első lépésben felhasználónév - jelszó alapú bejelentkezéssel. Ezután új modult hoztam létre az IDX szerverben, amely személyi igazolványszám kézzel történő begépelése segítségével azonosít.
\item Következő lépés az authentikáló mobilalkamazás létrehozása volt annak céljából, hogy a kézzel történő igazolványszám begépelése helyett a telefon beolvassa az elektronikus személyi számát, majd továbbítsa. Ehhez egy újabb tesztalkalmazást hoztam ltére, ahol a JMRTD nyílt forráskódú megoldás segítségével kiolvastam a szükséges adatot.
\item Létrehoztam a végleges mobilalkalmazás vázát egy újabb projektben.
\item Linket hoztam létre a bejelentkezési felületen, amely a szükséges paraméterek átadásával megnyitotta a végleges mobilalkalmazást, majd beolvasta a személyi igazolvány számát NFC segítségével. A továbbításhoz azonban egy authentikáló webalkalmazás létrehozása vált szükségessé, mivel az IDX szerver mobilról nem tudta az újabb kérést fogadni majd társítani az előző bejelentkezési kérelemhez.
\item Létrehoztam az authentikáló alkalmazást, és az IDX szerver új modulját átszerkesztettem úgy, hogy bejelentkezési kérelem esetén adatot továbbítson ennek az újonnan létrehozott alkalmazásnak. A mobiltelefonról pedig szintén ide küldtem az azonosítóval együtt a személyi igazolvány számát, validáltam az egyezőséget, majd az IDX szerver felületére kihelyezett gombra kattintva elfogadtam vagy elutasítottam a bejelentkezési kérelmet. Ezzel már csak a kihívás - válasz implementálása maradt hátra.
\item A tesztalkalmazásból a videófeldolgozást átemeltem a végleges mobilalkalmazásba, majd a bejelentkezés második lépéseként elindítottam a kihívás - válasz felületét. Az authentikáló alkalmazást átszerkesztettem úgy, hogy fogadja ezeket a képeket, társítsa, mentse le adatbázisba, majd a telefonról érkezett nyugta után továbbítsa a képeket a külső szereplőnek, a választ pedig dolgozza fel, majd jelezze, hogy elfogadta vagy elutasította a kérelmet. Prioritás volt az új külső szereplő könnyű integrációja a rendszerbe, erről írtam részletesebben a megvalósítás fejezetében. A megoldás működött, de nagyon lassú volt.
\item Gyorsítottam a megoldáson. A mobilalkalmazásban történő előfeldolgozást párhuzamosítottam, a képek felküldését szintén. Az authentikáló alkalmazáson adatbázis helyett memóriában tároltam a képeket, majd a kérés végén ezeket töröltem. A külső félnek történő képtovábbítást is párhuzamosítottam a Microsoft Project Oxford esetében, ám ugyan hivatalosan ezt nem tiltják, a feldolgozás sebessége drasztikusan csökkent ebben az esetben, így végül maradt a szekvenciális megvalósítás. Az alkalmazás működött, és lényegesen gyorsabb lett.
\item Különböző tömörítettségű képekre teszteltem az egész megoldás gyorsaságát és sikerességét, ez alapján megállapítottam a képek tömörítésének mértékét. A szükséges megbízhatóság mellett minimalizálni próbáltam ezzel a feldolgozási időt illetve a telefonról történő adatmennyiség továbbítását.
\item A kihívásra kapott választ feldolgozó algoritmust a később leírt módon optimalizáltam, szem előtt tartva a gyorsaságot, és megbízhatóságot (főleg az arcösszehasonlítás céljából felküldött kép minél kisebb elfordulására koncentrálva).
\item Hibakezeléseket vezettem be mind a webes felületre, mind a telefonra, szem előtt tartva azt, hogy az alkalmazások ne omolhassanak össze, és a felhasználó a lehető legtöbb visszajelzést kapja hiba esetén is, és egyértelmű legyen következő alkalommal a hiba elkerülésének módja.
\end{itemize}
\newpage